\documentclass[10pt,letterpaper,onecolumn]{article}
\usepackage{amsmath}
\usepackage{graphicx} 
\usepackage{hyperref}
\usepackage{subfig}
\graphicspath{ {img/} }
\begin{document}
\bibliographystyle{unsrt}
\title{Measuring the Lifetime of a Muon via Double Scintillator Detection\\
\large{Progress Report}}

\author{
 Akhil Deshpande \\*
 Nirmal Patel \\*
 \\*
 PHY 474 Advanced Laboratory \\*
 Spring 2024 \\*
 Dr. Deepa Thomas \\*
 Department of Physics \\*
 The University of Texas at Austin \\*
 Austin, TX 78712, USA
}
\date{\today}
\maketitle
\begin{abstract}
    This experiment utilizes a number of photomultiplier tubes and scintillators to measure the lifetime of muons, elementary particles generated by cosmic rays interacting with Earth's atmosphere. We collected data by detecting Cherenkov radiation within a lead glass box, as well as detecting the Cherenkov radiation of the muon's decay components. We are currently in the process of collecting and analyzing our data.
\end{abstract}
\section{Introduction}
\subsection{Historical Context}
The discovery of the muon in the 20th century marked a pivotal moment in the development of particle physics. Initially observed in cosmic ray experiments by Carl D. Anderson and Seth Neddermeyer at the California Institute of Technology in 1936, \cite{StreetStevenson:1937} the muon ($\mu$) was first mistaken for Yukawa's predicted particle responsible for the nuclear force. Yukawa's meson, theorized in 1935, was expected to mediate the strong interactions within the atomic nucleus, possessing a mass between that of the electron and the proton. However, the properties of the muon, once further analyzed, did not conform to those anticipated for the mediator of the nuclear force. \cite{Yukawa1935}


The muon is a lepton, a class of elementary particles not subject to the strong nuclear force, with a negative electric charge and a mass approximately 207 times that of the electron. \cite{CODATA2018MuonElectronMassRatio} The existence of the muon necessitated a significant expansion of the particle zoo, heralding the beginning of an era that would eventually lead to the establishment of the Standard Model of particle physics.


The clarification of the muon's nature and its differentiation from Yukawa's meson, later identified as the pion ($\pi$), was instrumental in the advancement of quantum field theories. This distinction emphasized the need for a comprehensive framework to understand the plethora of particles being discovered. The subsequent development of quantum electrodynamics (QED) and the electroweak theory, which unified the electromagnetic and weak forces, were partially motivated by the necessity to incorporate the muon and other leptons in a coherent theoretical structure. \cite{MuonG2Experiment2004}
\subsection{Theoretical Background}

The muon ($\mu$) is an elementary particle in the lepton family, akin to the electron but with a significantly greater mass. It carries a charge of $-1e$ and has a spin of $\frac{1}{2}$, classifying it as a fermion under the Standard Model of particle physics. The muon's mass is approximately $105.7 \, \text{MeV}/c^2$, making it about 207 times more massive than the electron. Despite its greater mass, the muon decays to an electron, a neutrino, and an antineutrino, due to the weak interaction, showcasing its unstable nature.

\subsubsection{Muon Decay}

The primary decay mode of the muon is represented by the following equation:
\[
\mu^- \rightarrow e^- + \bar{\nu}_e + \nu_\mu
\]
This decay process is mediated by the weak force, specifically through the exchange of a $W^-$ boson in the virtual state. The Feynman diagram for this decay illustrates the muon decaying into a muon neutrino and a virtual $W^-$ boson, which then converts into an electron and an electron antineutrino. The half-life of the muon in its rest frame is approximately $2.2 \, \mu\text{s}$, a duration that, while brief, allows for extensive experimental study of its decay processes and the verification of the Standard Model predictions. \cite{Beringer2012Leptons}

\subsubsection{Cherenkov Radiation}

When a muon travels through a medium at a speed greater than the phase velocity of light in that medium, it emits Cherenkov radiation, a phenomenon analogous to a sonic boom but with electromagnetic waves. The radiation is emitted at a characteristic angle $\theta$ relative to the direction of the particle's motion, which can be described by the formula:
\[
\cos\theta = \frac{1}{n\beta}
\]
where $n$ is the refractive index of the medium and $\beta$ is the speed of the particle relative to the speed of light in vacuum. 
\cite{Jackson1999Electrodynamics}

\section{Experimental Procedure}
\subsection{Apparatus}
\subsubsection{Overview}
This experiment focuses on detecting the presence of muons through the use of scinillators and photmultiplier tubes. We first set up a plastic scintillator on top of a lead glass box. This top scintillator will detect when a muon first passes through it. Then, the lead glass box will capture the muon, slowing it to rest. The muon will then decay, as shown above. The decay components will be measured by either a left of a right photomultiplier tube (PMT). When these PMTs activate, these signals are timed by our ORTEC setup, and we can achieve a spectrum of the lifetimes of the captured particles. 

An alternate setup we can consider is in the case of a single PMT and scintillator setup. We can wire this setup to a time-to-amplitude (TAC)converter directly. When the muon enters the scintillator, it will activate the PMT, and when it decays, it will again activate the PMT. This difference in times is measured by the TAC, and sorted into 'bins' by a multi-channel analyzer (MCA), such that it again provides us with an exponential decay spectrum.
\subsubsection{Scintillator}
\subsubsection{ORTEC Components}
\subsubsection{Computer Software}
\subsection{Data Collection}
We are currently in the process of collecting data. This involves letting our setup run for long periods of time, and observing the spectrum of counts we achieve from our MCA.
\section{Data Analysis}
\subsection{Calibration}
\subsection{Error and Uncertainty}
\section{Results and Conclusions}
\subsection{Results}
\subsection{Conclusion}
\paragraph*{Acknowledgments}
I would like to thank my lab partner, Nirmal Patel for his assistance on data collection. Furthermore, I'd like to thank Dr. Deepa Thomas, and Matthew Dwyer for their assistance throughout the measurement and set up processes. 
\newpage
\bibliography{refs} 

\end{document}

